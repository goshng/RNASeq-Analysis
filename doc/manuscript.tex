\documentclass{article}
\usepackage{amsmath}
\usepackage{amssymb}
\usepackage{graphicx}
\usepackage{color} 
\usepackage{setspace} 
\usepackage{rotating}
\usepackage{natbib}
\usepackage{xr}
\usepackage{../../latex/sty/genres}

% Text layout
\oddsidemargin 0in
\evensidemargin 0in
\topmargin -.5in
\textwidth 6.5in
\textheight 9in
\usepackage[labelfont=bf,labelsep=period,justification=raggedright]{caption}

\newcommand{\Rpackage}[1]{R package \textit{#1}}
\newcommand{\program}[1]{{\texttt{#1}}}
\newcommand{\fixme}[1]{{\textbf{Fixme:} \textit{\textcolor{blue}{#1}}}}

\makeatother
\pagestyle{myheadings}
\markright{Transcriptome of Streptococcus mutans}

\begin{document}

\begin{titlepage}

\title{Transcriptome \\
in {\em Streptococcus mutans} }

\author{Sang Chul Choi$^{1}$, Charles D.\ Danko$^{1}$, 
Melissa J.\ Hubisz$^{1}$, \\
Robert A.\ Burne$^{3}$
Michael J.\ Stanhope$^{2}$, and Adam Siepel$^{1}$}

\date{ }
\maketitle

\begin{footnotesize}
\begin{center}
$^1$Department of Biological Statistics and Computational Biology,\\
Cornell University, Ithaca, NY 14853, USA
\\[1ex]
$^2$Department of Population Medicine and Diagnostic Sciences,\\
College of Veterinary Medicine, Cornell University, Ithaca, NY 14853, USA
\\[1ex]
$^3$Department of Oral Biology,\\
College of Dentistry, University of Florida, Gainesville, Florida 32610 USA
\\
\end{center}
\end{footnotesize}

\vspace{1in}

\begin{tabular}{lp{4.5in}}
{\bf Submission type:}& Research Article
\vspace{1ex}\\
{\bf Keywords:}&Carbon catabolite repression, 
{\em Streptococcus mutans}
\vspace{1ex}\\
{\bf Running Head:}&Transcriptome of {\em Streptococcus mutans}
\vspace{1ex}\\ 
{\bf Corresponding Author:}&
\begin{minipage}[t]{4in}
 Adam Siepel\\
 102E Weill Hall, Cornell University\\
 Ithaca, NY 14853\\
 Phone: +1-607-254-1157\\
 Fax: +1-607-255-4698\\
 Email: acs4@cornell.edu
\end{minipage}
\end{tabular}

\thispagestyle{empty}
\end{titlepage}

\doublespacing

\section{Prewriting about UA159 versus TW1}

\subsection{Abstract}
A bacterial transcriptome of the primary etiological agent of human dental caries,
\textit{Stretptococcus mutans}, is described using RNA-seq data. Putative
transcripts boundaries are defined. Small RNAs are predicted by computational
methods, and confirmed by RNA-seq and Southern blot experiments. We refine the
previously established importance of CcpA regulation as central metabolism and
virulence gene expression using RNA-seq data.
CcpA is known to play a direct role in carbon catabolite repression (CCR).
We describe genes that are differentially expressed in glucose-grown wild-type
(UA159) and CcpA-deficient (TW1) strains. We found that more
genes were differentially expressed in the comparison between wild-type and
CcpA-deficient strains in galactose-grown strains than in
glucose-grown strains. How can I explain the reason?
We also found that more genes differentially expressed in wild-type than in
CcpA-deficient strain when comparing glucose- and galactose-grown cells. These
two findings were different from those of \cite{Abranches2008}. \fixme{We may
want to see effects of different p-value cutoff.} We confirmed that CcpA
modulates the pathogenic potential of \textit{S.\ mutans} through global control
of gene expression.
We predicted a few small RNAs based on sequence alignments and RNA secondary
structures. We confirmed them experimentally. Target genes of predicted small
RNAs were predicted. The target genes were enriched in some functional
categories. We listed highly potential small RNAs with their target genes
functional categories. 
Transcripts were predicted using RNA-seq short reads mapped on the wild-type
UA159 genome. 

\subsection{Reviews of papers that cited \cite{Abranches2008}}

Bacteria can selectively use substrates from different carbon sources. In the
presence of a
preferred carbon source, catabolic system of secondary substrates are often
regulated or shutdown: called carbon catabolite repression (CCR)
\cite{Gorke2008}. CCR also regulates the expression of virulence factors in many
pathogenic bacteria. A classical example is the glucose-lactose diauxie in
\textit{E.\ coli}. A significant number of genes can be subject to CCR.  
CCR in bacteria involves regulatory networks that activate or silence genes in
response to carbon source and availability.
\textit{S.\ mutans} possesses redundant systems for CCR.
We could measure directly the expression levels of \textit{fruA} in \textit{S.\
mutans}. 
We assess the contribution of CcpA to CCR of predicted operons.
Clearly, deletion of \textit{ccpA} resulted in substantial relief of the CCR of
\textit{fruA}
expression exerted by both glucose and fructose.


\subsection{Goal: Refinement of \cite{Abranches2008}}

Comparisons of microarray and RNA-seq are not good way to go.

Find CCR-sensitive transcripts.

\subsection{Methods and Materials}

\subsubsection{Bacterial strains and growth conditions}
\textit{S.\ mutans} UA159 is a human cariogenic isolate. The whole genome is
availabe at NCBI \cite{Ajdic2002}. Strain TW1 was derived from UA159 by
replacing a part of \textit{ccpA} gene \cite{Wen2002}. 

\textit{S.\ mutans} strains were inoculated into TV medium supplemented with
0.5\% galactose or glucose and grown to early exponential phase (OD600, 0.2).
Then fructose was added to the medium at concentrations (0,0.05,3,5,10 mM) for
3.5h to induce \textit{fruA} expression.

\subsubsection{RNA isolation}
RNA was isolated from S. mutans (1) by using 50 ml cultures that were grown
under the desired conditions and harvested by centrifugation at $4^{\circ}$C. Pelleted
cells were resuspended in 400 ul of diethyl pyrocarbonate-treated water, 800 ul
of RNAprotect reagent (Qiagen, Inc., Chatsworth, CA) was added, and the samples
were incubated at room temperature for 5 min with vortexing for 10 s at 1-min
intervals. Cells were then pelleted, resuspended in 250 ul Tris-EDTA (50:10)
buffer, and transferred to 1.5 ml screw-cap tubes containing sterile glass beads
(average diameter, 0.1 mm; Biospec, Bartlesville, OK), 10 ul of 1\% sodium
dodecyl sulfate, and 300 ul of acid phenol-chloroform (5:1). After
centrifugation for 15 min at maximum speed at $4^{\circ}$C, RNA was further purified
using an RNeasy mini kit (Qiagen, Inc., Chatsworth, CA), including on-column
DNase digestion with RNase-free DNase (Qiagen), as recommended by the supplier.

\subsubsection{RNA sequencing}
Sequencing libraries for the Illumina GA platform were constructed by shearing
the enriched cDNA by nebulisation (35psi, 6 min) followed by end-repair with
Klenow polymerase, T4 DNA polymerase and T4 polynucleotide kinase (to blunt-end
the DNA fragments). A single 3' adenosine moiety was added to the cDNA using
Klenow exo- and dATP. The Illumina adapters (containing primer sites for
sequencing and flowcell surface annealing) were ligated onto the repaired ends
on the cDNA and gel-electrophoresis was used to separate library DNA fragments
from unligated adapters by selecting cDNA fragments between 200-250 bps in size.
Fragmentation followed by gel-electrophoresis were used to separate library DNA
fragments and size fragments were recovered following gel extraction at room
temperature to ensure representation of AT rich sequences. Ligated cDNA
fragments were recovered following gel extraction at room temperature to ensure
representation of AT rich sequences. Libraries were amplified by 18 cycles of
PCR with Phusion polymerase. Sequencing libraries were denatured with sodium
hydroxide and diluted to 3.5 pM in hybridisation buffer for loading onto a
single lane of an Illumina GA flowcell. Cluster formation, primer hybridisation
and single-end, 36 cycle sequencing were performed using proprietary reagents
according to manufacturers' recommended protocol (https://icom.illumina.com/).
The efficacy of each stage of library construction was ascertained in a quality
control step that involved measuring the adapter-cDNA on a Agilent DNA 1000
chip. A final dilution of 2 nM of the library was loaded onto the sequencing
machine.

SOLiD sequencing was performed at Agencourt Bioscience (Beverly, MA) and
SeqWright (Houston, TX). Library preparations, fragment library protocols, and
all SOLiD-run parameters followed standard Applied Biosystems protocols.
Illumina/Solexa sequencing followed standard Illumina/Solexa methods and
protocols. Both systems were used because in this study our aim was simply to
collect as much sequence coverage as possible, and in pursuing this goal, we
utilized all available sequencing systems. We note, however, that the relative
performances (overall coverage, data output, etc.) noted in Table 1 for the
Illumina and Applied Biosystems sequencing platforms cannot be compared
directly. This study was not intended or designed to be a direct comparison of
the two systems, and interpreting the differences shown in Table 1 is difficult
given the many differences in cDNA/library preparation and the rapid evolution
of each system over the past year.


\subsubsection{Statistical analysis for differential expressions}
We used an R package, DESeq, \cite{Anders2010} to determine differentially
expressed genes based on negative binomial model originally developed by \cite{Robinson2007}.
Steps of analyzing RNA-seq data for determining differentially expressed genes
are described in \cite{Oshlack2010}. We used a short read aligner called
\program{BWA} \cite{Li2009a} to map the sequenced short reads on a reference genome,
\textit{Streptococcus mutans} UA159 (NCBI accession NC\_004350). Because the
aligner allowed a few gaps for efficient alignment of millions of sequences of
size being 100 base pairs long, short reads that consist of RNAs and sequencing
adapter sequences would be not mapped. We used \program{cutadapt} \cite{Martin2011} to
remove parts of adapter sequences from short reads before mapping on the
reference genome. Mapped short read alignments were converted into readable
formats \cite{Li2009b}. We counted short reads aligned to annotated genes in the
reference genome, forming a table of read counts of the genes. Statistical
software R \cite{R2011} with \Rpackage{DEseq} \cite{Anders2010}
was subsequently employed to infer differentially expressed in different
biological conditions.  We also generated tracks for our recently released
Streptococcus Genome Browser (http://strep-genome.bscb.cornell.edu) that
summarize gene expression levels based the short read alignments. This allowed
us to visualize parts of the reference genome for comparing different RNA
samples. Further analyses such as functional category enrichment test
\cite{Young2010} were also employed.

\subsubsection{small RNAs}

RNAz is used to find small RNA candidates. Sequence alignments were a
critical step for finding small RNAs in the approach. Phylogenies of streptococcus species could
be considered to choose species related with Streptococcus mutans.
Alternatively, I find sequences similar to intergenic regions, and refine the
alignments using a multiple sequence alignment tool such as MUSCLE
\cite{Edgar2004a}. Steps for
achieving this goal include: 1. Exract intergenic regions including up- and
down-stream, 2. BLAST the intergenic regions against non-redundant DB to find
similar sequences \cite{Altschul1990}, 3. Refine each alignment using MUSCLE \cite{Edgar2004a}, 4. Apply RNAz to the
alignments for scoring regions for small RNAs
\cite{Washietl2005,Gruber2010}, and 5. Find transcripts with high
score of small RNAs. The transcripts must be defined in advance.

RNAplex
\cite{Tafer2008}

A review paper about small RNA predictions and targets
\cite{Tafer2008a}

RNAplfold
\cite{Bernhart2006}

\subsubsection{Transcripts}

``-c 10 -b 25 -force\_gp''

\subsubsection{Gene category associations and virulence genes}

We assigned genes to Gene Ontology (GO) categories by comparing the {\em
  Streptococcus} genes to bacterial proteins from the Uniref90 database
using BLASTP, and then assigning the same GO classification as the target
gene of the uniProt GOA database if the match had an $E$-value of
$<1.0\times10^{-5}.$  
A gene family was assigned a given
classification if any of its genes was assigned that classification.  To
test for associations with replacing gene transfers we used the
recombination intensity estimated for each gene, and for additive gene
transfers we used the number of additive transfers inferred for each
family.  To test for significance, we performed a Mann-Whitney $U$ test of
the values (recombination intensities or numbers of transfers) associated
with a given category vs.\ the values for the all other genes/families.  We
used the \cite{Benjamini1995} method to correct for multiple comparisons.

\subsection{Results}

CcpA affects CCR of \textit{fruA}. 
\textit{fruA} fructan hydrolase
When Wen2002 deleted CRE sequencs in the \textit{fruA} promoter region, CCR was
almost completely alleviated. 
What are CCR-sensitive genes? 
What are the mechanism of CcpA on CCR of the \textit{fruA} operon? 
Expression levels of \textit{fruA} in \textit{S.\ mutans} strains that were
grown to early exponential phase (OD600, 0.2) in TV medium supplemented with
0.5\% glucose or galactose. \fixme{What does this mean?} 
Deletion of \textit{ccpA} resulted in substantial relief of the CCR of
\textit{fruA} expression exerted by both glucose and fructose. 
CcpA could directly interact with the \textit{fruA} promoter region. 

The role of CcpA in gene expression during growth on glucose. 
\textit{S.\ mutans} UA159 and TW1 grown under catabolite-repressing conditions
(glucose), and catabolite-derepressing conditions (galactose). Under
glucose-only condition I found xx number of differentially expressed genes. What
are the functional categories of those differentially expressed genes? What are
the genes that are most profoundly up- or down-regulated (Smu.1421 to Smu.1424,
PDH enzyme complex), ... and more? ... This regulatory control was
short-circuited in the strain growing on glucose and lacking CcpA. I showed that
some of KEGG pathways were affected by TW1 grown under glucose. We found
down-regulated genes, which would have something to do with pathogenicity. 
Upto here, I compared UA159 and TW1 under glucose.

1. UA159 (A) vs. TW1 (B) : fold change is B/A.
When glucose was the sole carbohydrate for growth, 170 genes were differentially
expressed at least twofold as a result of loss of CcpA (Fig. 2; see Table S1 in
the supplemental material). Of the gene products, 22 were predicted to partici-
pate in energy metabolism, 9 were PTS components, and 43 were hypothetical
proteins (Fig. 2). Among the most pro- foundly up-regulated genes in the ccpA
mutant strain were the genes that encoded the components of the PDH enzyme com-
plex (Smu.1421 to Smu.1424), pyruvate-formate lyase, enzymes in the
tricarboxylic acid (TCA) cycle, selected sugar-specific PTS permeases, and
components of the glycogen biosynthetic machinery (see Table S1 in the
supplemental material). 
Con- sistent with the profile data, TW1 cells grown in glucose had 78% higher
PDH activity than the wild-type strain (data not shown). Given the impact on
expression of PDH, TCA en- zymes, and selected glycolytic enzymes, CcpA clearly
legislates gene expression to discriminate between carbohydrate-replete and
carbohydrate-limiting conditions. In particular, when there is excess
carbohydrate, S. mutans moves carbohydrates primar- ily through lactate
dehydrogenase, sacrificing an ATP and gen- erating NAD􏰜. When carbohydrate is
limiting, S. mutans pro- duces mainly formic and acetic acids, shunts
carbohydrate through pyruvate formate lyase, and uses its partial TCA cycle to
regenerate reducing equivalents to maintain an appropriate NAD/NADH balance.
This regulatory control was clearly short-circuited in the strain growing on
glucose and lacking CcpA.
Among the more interesting genes that were down-regu- lated, significant
decreases in expression were noted for the genes encoding components of the
proton-translocating F-ATPase, phosphoglycerate kinase (Pgk), and
fructosyltrans- ferase (ftf), which catalyzes the conversion of sucrose to high-
molecular-weight polymers of fructose, as well as gtfD, which encodes the
glucosyltransferase D enzyme that produces pre- dominantly 􏰐-1,6-linked polymers
of glucose from sucrose. Since acid tolerance (ATPase), acid production from
carbohy- drate (Pgk), and glucan and fructan production (Gtf and Ftf), as well
as the ability to transport carbohydrates and produce intracellular storage
compounds (glycogen), as noted above (PTS and Glg), are central to the
pathogenic potential of S. mutans, CcpA represents a major control point for
expression of critical virulence attributes in cells presented with glucose.


Growth in derepressing conditions induces major changes in the UA159
transcriptome. Here, I compared glucose and galactose conditions for growing
UA159. Energy metabolism was enriched with differentially expressed genes. There
were more up-regulated genes in galactose condition compared with glucose
condition. Some genes are described in detail. 

2. I compared glucose (A) and galactose (B) for growing UA159
It is noteworthy that only three genes in the energy metabolism category were
down-regulated, whereas 23 genes in this category were markedly up-regulated. In
addition, the expression of 28 genes encoding hypothetical proteins was affected,
and 19 of them were up-regulated. The genes encod- ing proteins belonging to the
pathways for lactose and galac- tose utilization (2) were the most strongly
induced genes in cells growing on galactose. The galKTE genes, encoding the
enzymes for galac- tose utilization via the Leloir pathway (2), were
up-regulated between 10- and 36-fold, but the genes of the lactose (lac) operon,
encoding the lactose PTS permease and the tagatose- 6-phosphate pathway, were
up-regulated 70- to 300-fold. Since the products of the galKTE operon contribute
to the genera- tion of UDP-galactose, which is required for cell wall biogen-
esis, the lower level of induction of galKTE than of the lac operon may reflect
a need for higher constitutive levels of expression of the Leloir pathway in the
absence of galactose. Components of the multiple-sugar metabolism operon (msm)
were also induced in cells growing on galactose. The Msm pathway is induced by
raffinose and melibiose, both of which are ?-galactosides (5). The enhanced
expression of gene products involved in en- ergy metabolism, including pyruvate
formate lyase and com- ponents of the TCA cycle, in cells growing on galactose
is consistent with the idea that the cells perceive carbohydrate limitation. 

Now, I compared glucose (A) and galactose (B) for growing TW1. 

Now, I compared UA159 and TW1 under galactose. 

What genes are differentially expressed? 

Name the transcripts.
I found xx transcripts. Among them xx \% were mono-cistronic operons. xx \% were
poly-cistronic operons. xx transcripts could not defined because RNA-seq short
reads were not precise. Transripts were assigned with functional categories, and
differentially expressed transcripts were found. Transcripts and functional
categories: how can I assess that transcripts consist of genes with similar
functional cateogires. 

Name the small RNAs.
RNAz predicted a few small RNAs. Some of them were differentially expressed.
Some functional categories were enriched with target genes of the predicted
genes. 

\subsection{Discussion}
We used RNA-seq data to delineate transcript boundaries. Although this map of
transcripts does not precisely tell us where transcripts start and end, it
informs us of which open reading frames can be transcribed together. Potentially
polycistronic operons are defined. We focused on interpretations of operon in
light of CcpA regulations. 


\subsection{Abbreviations}

\begin{enumerate}
\item CCR Carbon catabolite repression
\item CcpA ?
\item IPS intracellular polysaccharide
\item CRE catabolite responsive elements
\end{enumerate}

\renewcommand*{\refname}{Literature Cited}
\bibliographystyle{../../latex/bst/mbe}
\bibliography{siepel-mutans}

\end{document}
